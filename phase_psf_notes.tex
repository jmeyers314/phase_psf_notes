\documentclass{article}
\usepackage[utf8]{inputenc}
\usepackage{geometry}
\usepackage[usenames,dvipsnames,svgnames,table]{xcolor}
%\usepackage{color}
\usepackage{graphicx}
\usepackage{amsmath}
\usepackage{bm}
\usepackage{amsfonts}
\usepackage{mathabx}
\usepackage{commath}

\geometry{textwidth=6.5in, textheight=9.0in,
    marginparsep=7pt, marginparwidth=.6in}
\setlength{\parindent}{0in}
\setlength{\parskip}{0.08in}

\newcommand{\red}[1]{\textcolor{red}{#1}}
\newcommand{\blue}[1]{\textcolor{blue}{#1}}
\newcommand*{\annot}[1]{\tag*{\footnotesize{\textcolor{gray}{#1}}}}
\newcommand{\like}{\mathcal{L}}
\let\Pr\undefined
\DeclareMathOperator{\Pr}{Pr}

\title{Notes on Phase Screen PSFs}
\author{Josh Meyers}
\date{October 2016}

\begin{document}

\section{Fluctuation statistics}

Mostly just going to steal the derivation in Roddier (1981), but try to make it self-comprehensible.

Start with the 1D temperature power spectrum $\Phi_T(\kappa)$ of temperature fluctuations $\Theta =
T - \langle T \rangle$:

\begin{equation}
    \Phi_T(\kappa)\, \propto\, \kappa^{-5/3}
\end{equation}

The (isotropic) 3D power spectrum is related to the 1D power spectrum by an integral over angles:
\begin{equation}
    \Phi_T(\kappa) = 4 \pi \kappa^2 \Phi_T(\bm{\kappa})
\end{equation}

Therefore

\begin{equation}
    \Phi_T(\bm{\kappa})\, \propto\, \kappa^{-11/3}
\end{equation}

The covariance of temperature fluctuations $B_T(\bm\rho) = \langle \Theta(\bm{r})
\Theta(\bm{r}+\bm\rho)\rangle$ is, by Wiener-Khinchin, the 3D Fourier transform of the power
spectrum:

\begin{equation}
    B_T(\bm\rho) = \int \Phi_T(\bm{\kappa}) \exp(i \bm\kappa \cdot \bm\rho) \dif\bm\kappa
\end{equation}

Unfortunately, that integral diverges at the origin, so it's instructive to instead consider the
structure function defined

\begin{equation}
    D_T(\bm\rho) = \langle | \Theta(\bm{r} + \bm\rho) - \Theta(\bm{r})|^2\rangle
\end{equation}

which is related to the covariance by

\begin{equation}
    D_T(\bm\rho) = 2\left[B_T(0) - B_T(\bm\rho)\right]
\end{equation}

Can use dimensional considerations to show that $D_T(\bm\rho) \propto \rho^{2/3}$.  Write

\begin{equation}
    D_T(\bm\rho) = C^2_T \rho^{2/3}
\end{equation}

Then can show (Tatarski (1961)) that

\begin{equation}
    \Phi_T(\bm{\kappa}) = \frac{\Gamma(\frac{8}{3})\sin(\pi/3)}{4 \pi^2} C^2_T \kappa^{-11/3}
\end{equation}

The same math essentially applies to the power spectrum of refractive index fluctuations.

\begin{equation}
    \Phi_N(\bm{\kappa}) = \frac{\Gamma(\frac{8}{3})\sin(\pi/3)}{4 \pi^2} C^2_N \kappa^{-11/3}
\end{equation}

\end{document}
